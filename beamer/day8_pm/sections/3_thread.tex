\section{<thread>}

\subsection{Basic Thread Management}

\begin{frame}[fragile]{Launching a Thread}
	\textbf{C++ Thread Constructor:}
	\begin{minted}{cpp}
template< class F, class... Args >
    explicit thread( F&& f, Args&&... args );
thread( thread&& other ) noexcept; // Move constructor
    \end{minted}
	\begin{itemize}
		\item \texttt{other}: another thread object to construct this thread object with
		\item \texttt{f}: Callable object to execute in the new thread
		\item \texttt{args}: Arguments to pass to the new function
	\end{itemize}
\end{frame}

\begin{frame}[fragile]{Launching a Thread - Basic Examples}
	\begin{columns}
		\begin{column}{0.5\textwidth}
			\textbf{Simple Function}
			\begin{minted}{cpp}
#include <thread>
#include <iostream>

void hello() {
    cout << "Hello from thread!"
              << endl;
}

int main() {
    thread t(hello);
    t.join();
    return 0;
}
			\end{minted}
		\end{column}
		\begin{column}{0.5\textwidth}
			\textbf{Function with Parameters}
			\begin{minted}{cpp}
#include <thread>
#include <iostream>

void worker(int id, const string& name) {
    cout << "Worker " << id
              << " (" << name << ") running"
              << endl;
}

int main() {
    thread t(worker, 42, "Alice");
    t.join();
    return 0;
}
			\end{minted}
		\end{column}
	\end{columns}
\end{frame}

\begin{frame}[fragile]{Launching a Thread - Function Objects and Lambdas}
	\begin{columns}
		\begin{column}{0.5\textwidth}
			\textbf{Function Object}
			\begin{minted}{cpp}
#include <thread>
#include <iostream>

class background_task {
public:
    void operator()() {
        cout << "Background task running"
                  << endl;
    }
};

int main() {
    background_task task;
    thread t(task);
    t.join();
    return 0;
}
			\end{minted}
		\end{column}
		\begin{column}{0.5\textwidth}
			\textbf{Lambda Expression}
			\begin{minted}{cpp}
#include <thread>
#include <iostream>

int main() {
    thread t([]() {
        cout << "Lambda task running"
                  << endl;
    });

    t.join();
    return 0;
}
			\end{minted}
		\end{column}
	\end{columns}
\end{frame}

\begin{frame}[fragile]{\emoji{warning} Most Vexing Parse Problem}
	\textbf{Common Pitfall:} C++ interprets ambiguous syntax as function declarations

	\begin{minted}{cpp}
struct background_task {
    void operator()() {
        // do work
    }
};

// WRONG! This declares a function, not creates a thread!
thread my_thread(background_task());

// Correct ways:
thread t1{background_task()};    // uniform initialization
thread t2((background_task()));  // extra parentheses
thread t3(background_task{});    // temporary object
	\end{minted}

	\textbf{Recommended:} Use uniform initialization \texttt{\{\}} or lambdas to avoid parsing issues
\end{frame}

\begin{frame}[fragile]{Waiting for a Thread to Complete}
	\textbf{join():} Wait for thread completion before continuing

	\begin{columns}
		\begin{column}{0.5\textwidth}
			\begin{minted}{cpp}
void worker(int seconds) {
    this_thread::sleep_for(
        chrono::seconds(seconds));
    cout << "Work done!" << endl;
}
int main() {
    cout << "Starting thread..." << endl;
    thread t(worker, 2);
    cout << "Waiting for completion..."
              << endl;
    t.join();  // Block until thread finishes
    cout << "Thread completed!" << endl;
    return 0;
}
			\end{minted}
			\textbf{Basic join() Example}
		\end{column}
		\begin{column}{0.5\textwidth}
			\begin{minted}{cpp}
void worker(int id) {
    cout << "Worker " << id
              << " running" << endl;
}
int main() {
    vector<thread> threads;
    for(int i = 0; i < 4; i++) {
        threads.emplace_back(worker, i);
    }
    for(auto& t : threads) {
        t.join();
    }
    cout << "All threads completed!"
              << endl;
    return 0;
}
			\end{minted}
			\textbf{Multiple Threads}
		\end{column}
	\end{columns}
\end{frame}

\begin{frame}[fragile]{Waiting in Exceptional Circumstances}
	\textbf{Problem:} Exception-safe thread management with RAII

	\begin{columns}
		\begin{column}{0.4\textwidth}
			\begin{minted}{cpp}
void do_work() {
    thread t([]() {
        // long running task
    });

    // If exception thrown here,
    // t.join() won't be called!
    risky_operation();

    t.join();  // May not be reached
}
			\end{minted}
			\textbf{Without RAII - Dangerous!}
		\end{column}
		\begin{column}{0.6\textwidth}
			\begin{minted}{cpp}
class thread_guard {
    thread& t;
public:
    explicit thread_guard(thread& t_) : t(t_) {}
    ~thread_guard() { if(t.joinable()) { t.join(); } }
    thread_guard(const thread_guard&) = delete;
    thread_guard& operator=(const thread_guard&) = delete;
};
void do_work() {
    thread t([]() { /* long running task */ });
    thread_guard g(t);
    risky_operation();  // Safe!
    // Destructor ensures join()
}
			\end{minted}
			\textbf{With RAII - Safe!}
		\end{column}
	\end{columns}
\end{frame}

\begin{frame}[fragile]{Running Threads in the Background}
	\textbf{detach():} Run thread independently without waiting

	\begin{columns}
		\begin{column}{0.5\textwidth}
			\begin{minted}{cpp}
    thread t(background_task);
    // Detach - thread runs independently
    t.detach();

    cout << "Main continues..." << endl;
    // Main might end before background finishes!
    this_thread::sleep_for(
        chrono::seconds(3));
			\end{minted}
			\textbf{Detached Thread Example}
		\end{column}
		\begin{column}{0.5\textwidth}
			\begin{minted}{cpp}
thread t(worker);
// Check if thread can be joined
if (t.joinable()) {
    cout << "Thread is joinable" << endl;
    t.join();
}
// After join/detach, not joinable
cout << t.joinable() << endl;  // false
thread t2(worker);
t2.detach();
cout << t2.joinable() << endl;  // false
			\end{minted}
			\textbf{Checking Thread State}
		\end{column}
	\end{columns}

	%\scriptsize
	\textbf{Important:} Every thread must be either joined or detached before destruction!
\end{frame}

\subsection{Passing Arguments to a Thread Function}

\begin{frame}[fragile]{Parameter Passing Basics}
	\textbf{Key Principle:} Thread constructor copies arguments as rvalues

	\textbf{String Conversion Issues}
	\begin{columns}
		\begin{column}{0.6\textwidth}
			\begin{minted}{cpp}
void print_string(const string& str) {
    cout << str << endl;
}

// UNSAFE! Pointer passed, conversion happens in new thread
thread unsafe_executor() {
    char buffer[] = "Hello World";
    return thread(print_string, buffer);
}
// SAFE! Force conversion before thread creation
thread safe_executor() {
    char buffer[] = "Hello World";
    return thread(print_string, string(buffer));
}
			\end{minted}
		\end{column}
		\begin{column}{0.4\textwidth}
			\begin{minted}{cpp}
try {
    thread t1 = unsafe_executor();
    t1.join();
} catch (const exception& e) {
    cerr << "Error in unsafe_executor: " << e.what() << endl;
}
thread t2 = safe_executor();
t2.join();
			\end{minted}
		\end{column}
	\end{columns}
\end{frame}

\begin{frame}[fragile]{Advanced Parameter Passing}

	\textbf{Member Functions}
	\begin{minted}{cpp}
class X {
public:
    void do_work(int n) {
        cout << "Work " << n
                  << " done by object" << endl;
    }
};

int main() {
    X my_x;

    // Pass member function and object
    thread t(&X::do_work, &my_x, 42);

    t.join();
    return 0;
}
			\end{minted}
\end{frame}

\subsection{Transferring Ownership of a Thread}

\begin{frame}[fragile]{Move Semantics with Threads}
	\textbf{thread is move-only:} Cannot be copied, only moved

	\begin{columns}
		\begin{column}{0.5\textwidth}
			\begin{minted}{cpp}
thread t1(worker);
// Move constructor
thread t2 = move(t1);
// t1 is now empty

// Move from function
thread t3 = create_thread();

// t1.join();  // ERROR! t1 is empty
t2.join();
t3.join();
			\end{minted}
			\textbf{Basic Move Operations}
		\end{column}
		\begin{column}{0.5\textwidth}
			\begin{minted}{cpp}
thread t1(task1);
thread t2(task2);

// Join first thread
t1.join();

// Move assign - t1 takes over t2
t1 = move(t2);
// t2 is now empty

t1.join();
			\end{minted}
			\textbf{Move Assignment}
		\end{column}
	\end{columns}
\end{frame}

\begin{frame}[fragile]{Transferring Ownership with Move - Advanced}
	\textbf{Move Semantics:} Transfer unique resources to threads

	\begin{minted}{cpp}
void process_data(unique_ptr<int> ptr) {
    if (ptr) {
        cout << "Processing: " << *ptr << endl;
        *ptr *= 2;
        cout << "Result: " << *ptr << endl;
    }
}

int main() {
    auto ptr = make_unique<int>(42);

    // Transfer ownership to thread using move
    thread t(process_data, move(ptr));

    // ptr is now nullptr, ownership transferred
    assert(ptr == nullptr);

    t.join();
    return 0;
}
	\end{minted}
\end{frame}

\begin{frame}[fragile]{Thread Containers and RAII}
	\textbf{Managing Multiple Threads with Move Semantics}

	\begin{columns}
		\begin{column}{0.5\textwidth}
			\begin{minted}{cpp}
class scoped_thread {
    thread t;
public:
    explicit scoped_thread(thread t_) : t(move(t_)) {
        if (!t.joinable()) {
            throw logic_error("No thread");
        }
    }

    ~scoped_thread() {
        t.join();
    }

    scoped_thread(const scoped_thread&) = delete;
    scoped_thread& operator=(const scoped_thread&) = delete;
};
            \end{minted}
		\end{column}
		\begin{column}{0.5\textwidth}
			\begin{minted}{cpp}
int main() {
    vector<scoped_thread> threads;

    for (int i = 0; i < 3; ++i) {
        threads.emplace_back(thread(worker, i));
    }

    // Threads automatically joined when scoped_thread destructors run
    return 0;
}
	\end{minted}
		\end{column}
	\end{columns}
\end{frame}

\subsection{Choosing the Number of Threads at Runtime}

\begin{frame}[fragile]{Hardware Concurrency}
	\textbf{thread::hardware\_concurrency():} Get optimal thread count

	\begin{minted}{cpp}
unsigned int num_threads = thread::hardware_concurrency();

if (num_threads == 0) {
    num_threads = 2;  // Fallback if not available
}

cout << "Using " << num_threads << " threads" << endl;

vector<thread> threads;
for (unsigned int i = 0; i < num_threads; ++i) {
    threads.emplace_back(worker, i);
}

for (auto& t : threads) {
    t.join();
}
	\end{minted}
\end{frame}

\begin{frame}[fragile]{Parallel Accumulate - Helper Function}
	\textbf{Step 1:} Define the worker function for each thread

	\begin{minted}{cpp}
template<typename Iterator, typename T>
void accumulate_block(Iterator first, Iterator last, T& result) {
    result = accumulate(first, last, result);
}
	\end{minted}

	\textbf{Function Purpose:}
	\begin{itemize}
		\item Each thread processes a block of data
		\item Computes sum of elements from \texttt{first} to \texttt{last}
		\item Stores result in reference parameter
		\item Uses standard library \texttt{accumulate} function
	\end{itemize}

	\vspace{1em}
	\textbf{Next:} Calculate optimal thread configuration
\end{frame}

\begin{frame}[fragile]{Parallel Accumulate - Thread Configuration}
	\textbf{Step 2:} Calculate optimal number of threads and work distribution

	\begin{minted}{cpp}
template<typename Iterator, typename T>
T parallel_accumulate(Iterator first, Iterator last, T init) {
    unsigned long const length = distance(first, last);
    if (!length) return init;

    // Configuration parameters
    unsigned long const min_per_thread = 25;
    unsigned long const max_threads =
        (length + min_per_thread - 1) / min_per_thread;

    unsigned long const hardware_threads =
        thread::hardware_concurrency();

    unsigned long const num_threads =
        min(hardware_threads != 0 ? hardware_threads : 2, max_threads);

    unsigned long const block_size = length / num_threads;
    // ...
}
	\end{minted}

	\textbf{Logic:} Balance between overhead and parallelism
\end{frame}

\begin{frame}[fragile]{Parallel Accumulate - Thread Creation \& Execution}
	\textbf{Step 3:} Create threads and distribute work

	\begin{minted}{cpp}
// Storage for results and threads
vector<T> results(num_threads);
vector<thread> threads(num_threads - 1);
// Create worker threads (all but last)
Iterator block_start = first;
for (unsigned long i = 0; i < (num_threads - 1); ++i) {
    Iterator block_end = block_start;
    advance(block_end, block_size);
    threads[i] = thread(accumulate_block<Iterator, T>,
                        block_start, block_end, ref(results[i]));
    block_start = block_end;
}
// Main thread handles the last block
accumulate_block(block_start, last, results[num_threads - 1]);
// Wait for all threads and combine results
for (auto& entry : threads) entry.join();
return accumulate(results.begin(), results.end(), init);
	\end{minted}
\end{frame}

\subsection{Identifying Threads}

\begin{frame}[fragile]{Thread IDs - Basic Usage}
	\textbf{thread::id:} Unique identifier for each thread

	\begin{minted}{cpp}
void worker() {
    cout << "Thread ID: " << this_thread::get_id() << endl;
}
cout << "Main thread ID: " << this_thread::get_id() << endl;
thread t(worker);
cout << "Worker thread ID: " << t.get_id() << endl;
t.join();
cout << "After join: " << t.get_id() << endl;  // Default ID
	\end{minted}

	\textbf{Key Functions:}
	\begin{itemize}
		\item \texttt{this\_thread::get\_id()}: Get current thread's ID
		\item \texttt{thread\_obj.get\_id()}: Get thread object's ID
		\item Default-constructed ID represents "no thread"
	\end{itemize}
\end{frame}

\begin{frame}[fragile]{Thread ID Properties \& Usage}
	\textbf{Thread ID Characteristics:}
	\begin{itemize}
		\item \textbf{Unique:} Each executing thread has a unique ID
		\item \textbf{Comparable:} Can be compared for equality and ordering
		\item \textbf{Hashable:} Can be used as keys in containers
		\item \textbf{Default-constructible:} Default ID represents "no thread"
	\end{itemize}

	\begin{minted}{cpp}
// Thread ID comparison
thread::id main_id = this_thread::get_id();
thread::id default_id;  // Default-constructed
cout << "IDs equal: " << (main_id == default_id) << endl;  // false
// Use as container key
unordered_set<thread::id> thread_set;
thread_set.insert(main_id);
// Useful for debugging and thread identification
map<thread::id, string> thread_names;
thread_names[this_thread::get_id()] = "main_thread";
	\end{minted}
\end{frame}

\begin{frame}[fragile]{\emoji{light-bulb} Best Practices Summary}
    \scriptsize
	\textbf{Thread Management:}
	\begin{itemize}
		\item Always \texttt{join()} or \texttt{detach()} threads before destruction
		\item Use RAII (e.g., \texttt{thread\_guard}) for exception safety
		\item Prefer \texttt{move} semantics for thread ownership transfer
	\end{itemize}

	\textbf{Parameter Passing:}
	\begin{itemize}
		\item Be careful with string literals - convert to \texttt{std::string} explicitly
		\item Use \texttt{std::ref()} for reference parameters
		\item Move unique resources (e.g., \texttt{unique\_ptr}) when transferring ownership
	\end{itemize}

	\textbf{Performance:}
	\begin{itemize}
		\item Use \texttt{hardware\_concurrency()} to determine optimal thread count
		\item Balance thread overhead vs. parallelism (minimum work per thread)
		\item Consider thread pool patterns for repeated tasks
	\end{itemize}
\end{frame}

\begin{frame}[fragile]{\emoji{books} Key Takeaways}
	\textbf{Core Concepts Mastered:}
	\begin{enumerate}
		\item \textbf{Thread Creation:} Functions, lambdas, member functions
		\item \textbf{Synchronization:} \texttt{join()} vs \texttt{detach()} patterns
		\item \textbf{Exception Safety:} RAII wrappers for automatic cleanup
		\item \textbf{Move Semantics:} Efficient thread and resource transfer
		\item \textbf{Runtime Optimization:} Hardware-aware thread configuration
	\end{enumerate}

	\vspace{1em}
	\textbf{Next Steps:}
	\begin{itemize}
		\item Thread synchronization with mutexes and locks
		\item Condition variables for thread coordination
		\item Parallel algorithms in the standard library
	\end{itemize}

	\vspace{1em}
	\textbf{Remember:} Modern C++ threading is about safety, efficiency, and expressiveness!
\end{frame}
