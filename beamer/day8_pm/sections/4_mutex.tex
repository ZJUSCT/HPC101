\section{<mutex>}

\subsection{Race Conditions}
\begin{frame}[fragile]{Race Condition Problem}
	\textbf{The Core Issue}: Multiple threads accessing shared data

	\begin{minted}{cpp}
int counter = 0;  // Shared variable

void increment_unsafe(int iterations) {
    for(int i = 0; i < iterations; i++) {
        counter++;  // NOT thread-safe!
    }
}
	\end{minted}

	\begin{minted}{cpp}
// 4 threads, each incrementing 100000 times
std::vector<std::thread> threads;
for(int i = 0; i < 4; i++) {
    threads.emplace_back(increment_unsafe, 100000);
}
for(auto& t : threads) { t.join(); }

// Expected: 400000, Actual: varies (e.g., 387234)
	\end{minted}
\end{frame}

\begin{frame}[fragile]{Why Race Conditions Happen}
	\textbf{Assembly breakdown of} \texttt{counter++}:

	\begin{columns}
		\begin{column}{0.4\textwidth}
			\begin{minted}{asm}
mov eax, [counter]  // Read
add eax, 1          // Modify
mov [counter], eax  // Write
			\end{minted}
		\end{column}
		\begin{column}{0.6\textwidth}
			\textbf{Thread Interleaving Problem}:
			\begin{itemize}
				\item Thread 1 reads counter = 0
				\item Thread 2 reads counter = 0
				\item Thread 1 writes counter = 1
				\item Thread 2 writes counter = 1
				\item \textbf{Lost update!} Should be 2
			\end{itemize}
		\end{column}
	\end{columns}

	\vspace{0.5em}
	\textbf{Critical Section}: Code accessing shared resources
\end{frame}

\subsection{Protecting Shared Data with Mutexes}
\begin{frame}[fragile]{C++ Mutex Solution}
	\textbf{Using} \texttt{std::mutex} \textbf{with RAII}:

	\begin{minted}{cpp}
std::mutex counter_mutex;
int counter = 0;

void increment_safe(int iterations) {
    for(int i = 0; i < iterations; i++) {
        std::lock_guard<std::mutex> lock(counter_mutex);
        counter++;  // Now thread-safe!
        // Automatic unlock when lock goes out of scope
    }
}
	\end{minted}

	\textbf{RAII Benefits}:
	\begin{itemize}
		\item Exception safety
		\item Automatic cleanup
		\item No manual lock/unlock
	\end{itemize}
\end{frame}

\begin{frame}[fragile]{Structured Data Protection}
	\textbf{Class-based Protection}:

	\begin{minted}{cpp}
class ThreadSafeCounter {
private:
    int count = 0;
    mutable std::mutex mtx;

public:
    void increment() {
        std::lock_guard<std::mutex> lock(mtx);
        ++count;
    }

    int get() const {
        std::lock_guard<std::mutex> lock(mtx);
        return count;
    }
};
	\end{minted}

	\textbf{Key Principle}: Keep data and mutex together
\end{frame}

\subsection{Deadlock Prevention}
\begin{frame}[fragile]{The Deadlock Problem}
	\textbf{Classic Deadlock Scenario}:

	\begin{minted}{cpp}
std::mutex mutex1, mutex2;

void thread1() {
    std::lock_guard<std::mutex> lock1(mutex1);
    // ... some work ...
    std::lock_guard<std::mutex> lock2(mutex2);  // Waits forever
}

void thread2() {
    std::lock_guard<std::mutex> lock2(mutex2);
    // ... some work ...
    std::lock_guard<std::mutex> lock1(mutex1);  // Waits forever
}
	\end{minted}

	\textbf{Result}: Circular wait → Deadlock!
\end{frame}

\begin{frame}[fragile]{Deadlock Solutions}
	\textbf{Strategy 1: Ordered Locking}
	\begin{minted}{cpp}
// Always lock in same order
std::lock_guard<std::mutex> lock1(mutex1);  // First
std::lock_guard<std::mutex> lock2(mutex2);  // Second
	\end{minted}

	\textbf{Strategy 2: Atomic Multi-lock}
	\begin{minted}{cpp}
std::unique_lock<std::mutex> lock1(mutex1, std::defer_lock);
std::unique_lock<std::mutex> lock2(mutex2, std::defer_lock);
std::lock(lock1, lock2);  // Atomic acquisition
	\end{minted}
\end{frame}

\begin{frame}[fragile]{Deadlock Solutions}
	\textbf{Strategy 3: Address-based Ordering}
	\begin{minted}{cpp}
if (&mutex1 < &mutex2) {
    std::lock_guard<std::mutex> lock1(mutex1);
    std::lock_guard<std::mutex> lock2(mutex2);
} else {
    std::lock_guard<std::mutex> lock2(mutex2);
    std::lock_guard<std::mutex> lock1(mutex1);
}
	\end{minted}
\end{frame}

\begin{frame}[fragile]{Practical Example: Bank Transfer}
	\begin{minted}{cpp}
class BankAccount {
    double balance;
    mutable std::mutex mtx;
public:
    static void transfer(BankAccount& from, BankAccount& to,
                        double amount) {
        // Prevent deadlock with consistent ordering
        if (&from < &to) {
            std::lock_guard<std::mutex> lock1(from.mtx);
            std::lock_guard<std::mutex> lock2(to.mtx);
        } else {
            std::lock_guard<std::mutex> lock1(to.mtx);
            std::lock_guard<std::mutex> lock2(from.mtx);
        }
        from.balance -= amount;
        to.balance += amount;
    }
};
	\end{minted}
\end{frame}

\subsection{Key Takeaways}
\begin{frame}{Mutex Best Practices}
	\begin{enumerate}
		\item \textbf{Use RAII}: \texttt{std::lock\_guard} over manual lock/unlock
		\item \textbf{Keep critical sections short}: Minimize lock duration
		\item \textbf{Consistent lock ordering}: Prevent deadlocks
		\item \textbf{Encapsulate}: Keep data and mutex together
		\item \textbf{Avoid nested locks}: When possible
	\end{enumerate}
\end{frame}

\begin{frame}{C++ Mutex Tools}
	\begin{itemize}
		\item \texttt{std::mutex} - Basic mutual exclusion
		\item \texttt{std::lock\_guard} - RAII lock wrapper
		\item \texttt{std::unique\_lock} - Flexible locking
		\item \texttt{std::lock()} - Multi-mutex atomic locking
	\end{itemize}
\end{frame}
